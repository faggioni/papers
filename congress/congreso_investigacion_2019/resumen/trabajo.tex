\documentclass[12pt]{book}
%OpenAny as a argunment on documentclass	, Avoid Chapters to start in odd pages
\usepackage[latin1]{inputenc}
\usepackage[spanish,es-noquoting]{babel}
\usepackage[eps]{graphicx}
\usepackage{titling}
\usepackage{vmargin}
\usepackage{url}
\usepackage{textcomp}
\usepackage[hidelinks]{hyperref}
\usepackage{tabularx}
\usepackage{adjustbox}
\usepackage{float}
\usepackage[T1]{fontenc}
\usepackage{epstopdf}
\usepackage{apacite}
\usepackage{amsmath}
\usepackage{amssymb}
\usepackage{amsthm}
\usepackage{multicol}
\usepackage{bm}
\usepackage{calc}
\usepackage{everypage}
\usepackage{adjustbox}
\usepackage{multirow}
\usepackage{lscape}
\usepackage{makecell}
\usepackage{rotating}
\usepackage[table,xcdraw]{xcolor}
\usepackage{tikz}
\usetikzlibrary{shapes,arrows}
\usepackage{pstricks-add}
\usepackage{pstricks}
\usepackage{fancyhdr}
\usepackage{setspace}
\usepackage{arydshln} %Dash Matrices
\usepackage{titlesec}
\usepackage{mathrsfs} %Laplace Transform L
\usepackage[toc,page]{appendix} %Apendix
\usepackage{verse}
%text in top of letters
\usepackage{stackrel}
%Symbols and Abbreviations Table
\usepackage{enumitem}
\newlist{abbrv}{itemize}{1}
\setlist[abbrv,1]{label=,labelwidth=1in,align=parleft,itemsep=0.1\baselineskip,leftmargin=!}
%Footnote in Tabular
\usepackage{tablefootnote}


\setpapersize{USletter}
%Setting Margins
\setmarginsrb{35mm}{35mm}{35mm}{30mm}{0pt}{0mm}{0pt}{0mm}

%Interlineado
\renewcommand{\baselinestretch}{1.5}

%Sangria
\usepackage{parskip}
\setlength{\parindent}{0.75cm}

%columns Separation
\setlength{\columnsep}{2cm}

%Fancy Header Setup
\rhead{}
\lhead{}
\chead{\bfseries \nouppercase{\leftmark}}

%Footer's Margins
\headsep = 30pt
\footskip = 25pt
\renewcommand{\headrulewidth}{0.5pt}


%Set 
\linespread{1.50}

\begin{document}

	\vspace{2cm}
	\addcontentsline{toc}{chapter}{Resumen}
	
	\begin{center}
		\textbf{Consideraciones �ptimas en el Dise�o de Controladores de dos grados de libertad mediante un enfoque polin�mico}
	\end{center}
	
	\vspace{1cm}
	\thispagestyle{empty}
	\begin{flushright} 
		Ponente: Miguel Faggioni \\
		Caracas, Octubre 2019
	\end{flushright}
	
	\begin{center}
		\textbf{RESUMEN}
	\end{center}
		Los controladores de dos grados de libertad han sido desarrollados dentro de las descripciones externas de los sistemas continuos y sus t�cnicas de dise�o son limitadas a metodolog�as algebraicas asociadas a la localizaci�n de polos. A su vez, estrategias de dise�o como el Regulador �ptimo Cuadr�tico hacen vida dentro del estudio de sistema de control cuya planta esta modelada bajo descripciones internas, en particular, espacios de estado. Estas ultimas abarcan un an�lisis mucho mas complejo pero a su vez permiten obtener un sistema de control mucho mas sofisticado.
		
		En la presente ponencia se establecen mecanismos de obtenci�n de controladores de dos grados de libertad bajo t�cnicas de dise�o �ptimas haciendo uso de modelos matem�ticos que emplean la informaci�n interna del sistema. Para esto se define una representaci�n en espacio de estados haciendo uso exclusivamente de la informaci�n medible del sistema, esto es, entrada y salida. Dicha representaci�n es el enlace que permite aplicar t�cnicas de control avanzadas a plantas que de otra manera serian tratadas con herramientas de an�lisis b�sicas.
		
		Esta propuesta permite obtener controladores descritos en el dominio frecuencial que heredan caracter�sticas optimas dentro de la din�mica transitoria propias del Regular Optimo Cuadr�tico, as� mismo, considera las posibles perturbaciones a la que pueda estar expuesta el sistema de control, esto aplicando el principio del modelo interno.
		
		Se presentara la metodolog�a de dise�o, as� como la verificaci�n de su funcionamiento ante plantas con distinta naturaleza f�sica, esto haciendo uso de simulaci�n bajo un ambiente computacional.
	 
		\thispagestyle{empty}
		\noindent \textbf{Palabras Claves}: LQR,  Controladores de dos Grados de Libertad, Control Robustos, Rechazo a Perturbaciones.
\end{document}